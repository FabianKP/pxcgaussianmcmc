
\documentclass[10pt]{article}

%~~~~~~~~~~
% preamble
%~~~~~~~~~~

%'
%' common packages
%'
\usepackage[utf8]{inputenc}
\usepackage[T1]{fontenc}

\usepackage{amsmath,amssymb,dsfont}
\numberwithin{equation}{section}
\usepackage{microtype}
\usepackage{graphicx,tikz,pgfplots}
\graphicspath{{images/}}
\pgfplotsset{compat=newest}
\usepackage[hyperref,amsmath,thmmarks]{ntheorem}
\usepackage{aliascnt}
\usepackage[a4paper,centering,bindingoffset=0cm,marginpar=2cm,margin=2.5cm]{geometry}
%\usepackage[pagestyles]{titlesec}
\usepackage[font=footnotesize,format=plain,labelfont=sc,textfont=sl,width=0.75\textwidth,labelsep=period]{caption}
\usepackage{bm}
\usepackage{bbm}

%'
%' biblatex
%'
\usepackage[backend=biber,maxnames=10,backref=true,hyperref=true,giveninits=true,safeinputenc]{biblatex}
\bibliography{proximal_mcmc.bib}


\DefineBibliographyStrings{english}{%
	backrefpage = {cited on page},
	backrefpages = {cited on pages},
}


\title{Proximal MCMC for linearly constrained multivariate normal distributions}
\author{FabianKP}
\date{}

%'
%' writes the title always in quotes.
%'
\DeclareFieldFormat[report]{title}{``#1''}
\DeclareFieldFormat[book]{title}{``#1''}
\AtEveryBibitem{\clearfield{url}}
\AtEveryBibitem{\clearfield{note}}


\usepackage[pdftex,colorlinks=true,linkcolor=blue,citecolor=green,urlcolor=blue,bookmarks=true,bookmarksnumbered=true]{hyperref}
\hypersetup
{
    pdfauthor={FabianKP},
    pdfsubject={Proximal MCMC for linearly constrained multivariate normal distributions},
    pdftitle={}
}
\def\sectionautorefname{Section}
\def\subsectionautorefname{Section}
\def\subsubsectionautorefname{Section}

\graphicspath{{images/}}

%
% --------------- ABBREVIATIONS
%

\include{abbreviations}


%
% --------------- BEGIN DOCUMENT
%


\begin{document}

\maketitle

\section*{Notation}

\begin{itemize}
\item $\indicator{C}$ denotes the indicator function,
\begin{align*}
\indicator{C}(x) := \begin{cases}
1, & \text{if } x \in C, \\
0, & \text{otherwise}.
\end{cases}
\end{align*}
\end{itemize}


\section{Introduction}

We consider linearly constrained multivariate normal distributions on $\R^d$. Such distributions have the general form

\begin{align*}
p(\bm x) & = (2 \pi)^{d/2} \det(\bm \Sigma)^{-1} \exp( - \frac{1}{2} (\bm x - \bm m)^\top \bm \Sigma^{-1} (\bm x - \bm m) \indicator{C}(\bm x), \\
C &= \Set{\bm x \in \R^d}{\bm A \bm x = \bm b, \bm C \bm x \geq \bm d, \bm l \leq \bm x \leq \bm u}.
\end{align*}

Sampling from this distribution is important in a range of applications, in particular in Bayesian inverse problems with constraints. In such applications, the dimension $d$ will often be very large, and the covariance matrix $\bm \Sigma$ will sometimes be ill-conditioned. In that case, sampling from $p$ using non-specialized methods does not work.

One method that was proposed for sampling from ill-conditioned log-concave distributions such as these is the so-called \textbf{proximal Markov chain Monte Carlo method} (proximal MCMC) \cite{Per16, DurMouPer18}.






\addcontentsline{toc}{section}{Bibliography}

\printbibliography

\end{document}
