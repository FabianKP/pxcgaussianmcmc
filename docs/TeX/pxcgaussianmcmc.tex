
\documentclass[10pt]{article}

%~~~~~~~~~~
% preamble
%~~~~~~~~~~

%'
%' common packages
%'
\usepackage[utf8]{inputenc}
\usepackage[T1]{fontenc}

\usepackage{amsmath,amssymb,dsfont}
\numberwithin{equation}{section}
\usepackage{microtype}
\usepackage{graphicx,tikz,pgfplots}
\graphicspath{{images/}}
\pgfplotsset{compat=newest}
\usepackage[hyperref,amsmath,thmmarks]{ntheorem}
\usepackage{aliascnt}
\usepackage[a4paper,centering,bindingoffset=0cm,marginpar=2cm,margin=2.5cm]{geometry}
%\usepackage[pagestyles]{titlesec}
\usepackage[font=footnotesize,format=plain,labelfont=sc,textfont=sl,width=0.75\textwidth,labelsep=period]{caption}
\usepackage{bm}
\usepackage{bbm}
\usepackage{algorithm, algpseudocode}

%'
%' biblatex
%'
\usepackage[backend=biber,maxnames=10,backref=true,hyperref=true,giveninits=true,safeinputenc]{biblatex}
\bibliography{proximal_mcmc.bib}


\DefineBibliographyStrings{english}{%
	backrefpage = {cited on page},
	backrefpages = {cited on pages},
}


\title{Proximal MCMC for linearly constrained multivariate normal distributions}
\author{FabianKP}
\date{}

%'
%' writes the title always in quotes.
%'
\DeclareFieldFormat[report]{title}{``#1''}
\DeclareFieldFormat[book]{title}{``#1''}
\AtEveryBibitem{\clearfield{url}}
\AtEveryBibitem{\clearfield{note}}


\usepackage[pdftex,colorlinks=true,linkcolor=blue,citecolor=green,urlcolor=blue,bookmarks=true,bookmarksnumbered=true]{hyperref}
\hypersetup
{
    pdfauthor={FabianKP},
    pdfsubject={Proximal MCMC for linearly constrained multivariate normal distributions},
    pdftitle={}
}
\def\sectionautorefname{Section}
\def\subsectionautorefname{Section}
\def\subsubsectionautorefname{Section}

\graphicspath{{images/}}

%
% --------------- ABBREVIATIONS
%

%'
%' common mathematical commands.
%'
\newtheorem{lemma}{Lemma}[section]
\def\lemmaautorefname{Lemma}

\newaliascnt{proposition}{lemma}
\newtheorem{proposition}[proposition]{Proposition}
\aliascntresetthe{proposition}
\def\propositionautorefname{Proposition}

\newaliascnt{corollary}{lemma}
\newtheorem{corollary}[corollary]{Corollary}
\aliascntresetthe{corollary}
\def\corollaryautorefname{Corollary}

\newaliascnt{theorem}{lemma}
\newtheorem{theorem}[theorem]{Theorem}
\aliascntresetthe{theorem}
\def\theoremautorefname{Theorem}

\theorembodyfont{\normalfont}
\newaliascnt{definition}{lemma}
\newtheorem{definition}[definition]{Definition}
\aliascntresetthe{definition}
\def\definitionautorefname{Definition}

\newaliascnt{assumption}{lemma}
\newtheorem{assumption}[assumption]{Assumption}
\aliascntresetthe{assumption}
\def\assumptionautorefname{Assumption}

\theoremstyle{nonumberplain}
\theoremseparator{:}
\theoremheaderfont{\normalfont\itshape}

\newtheorem{remark}{Remark}
\def\remarkautorefname{Assumption}

\newtheorem{example}{Example}
\def\remarkautorefname{Example}

%\theoremsymbol{\ensuremath{\square}}
%\newtheorem{proof}{Proof}

\newcommand{\N}{\mathds{N}}
\newcommand{\Z}{\mathds{Z}}
\newcommand{\R}{\mathds{R}}
\newcommand{\C}{\mathds{C}}

% real part
\let\RE\Re
\let\Re=\undefined
\DeclareMathOperator{\Re}{\RE e}
% imaginary part
\let\IM\Im
\let\Im=\undefined
\DeclareMathOperator{\Im}{\IM m}
% support
\DeclareMathOperator{\supp}{supp}
% sign
\DeclareMathOperator{\sign}{sign}
% absolute Value
\newcommand{\abs}[1]{\left|#1\right|}
% norm
\newcommand{\norm}[1]{\left\|#1\right\|}
% set
\newcommand{\set}[1]{\left\{#1\right\}}
% inner Product
\newcommand{\inner}[2]{\left<#1,#2\right>}
% euler number
\newcommand{\e}{\mathrm e}
% imaginary unit (command \i is stored in \ii)
\let\ii\i
\renewcommand{\i}{\mathrm i}
% derivative
\renewcommand{\d}{\,\mathrm d}

%'
%' comments
%'
\newcommand{\commentO}[1]{\textcolor{red}{#1}}
\newcommand{\commentF}[1]{\textcolor{blue}{#1}}


% Fabian's abbreviations

\DeclareMathOperator*{\esssup}{ess\,sup}
\DeclareMathOperator{\dist}{dist}
\DeclareMathOperator*{\argmin}{argmin}
\DeclareMathOperator*{\argmax}{argmax}

\newcommand{\1}{\mathbbm{1}}
\newcommand{\Set}[2]{\left\{\,#1:\, #2\,\right\}}
\newcommand{\dom}[1]{ \mathcal{D}(#1) }%domain of an unbounded operator
\newcommand{\ran}[1]{ \mathcal{R}(#1) }%range of a linear operator
\newcommand{\nullspace}[1]{ \mathcal{N}(#1) }%the null space of an operator

\newcommand{\linspan}{\mathrm{span}}
\newcommand{\powerset}{\mathfrak{P}}

\newcommand{\rank}{\mathrm{rank}}

% spaces
\newcommand{\Rplus}{\R^+}
\newcommand{\EE}{\mathbb{E}}
\newcommand{\HH}{\mathbb{H}}
\newcommand{\UU}{\mathbb{U}}
\newcommand{\VV}{\mathbb{V}}
\newcommand{\WW}{\mathbb{W}}
\newcommand{\XX}{\mathbb{X}}
\newcommand{\YY}{\mathbb{Y}}
\newcommand{\ZZ}{\mathbb{Z}}
\newcommand{\Lin}{\mathrm{Lin}} %the space of linear functions
\newcommand{\BL}{\mathcal{L}} % the space of bounded linear operators

% operators\\
\newcommand{\D}{\,\mathrm D}
\newcommand{\tr}{\mathrm{tr}}
\newcommand{\indicator}[1]{ {\mathbbm{1}_{#1}} }
\newcommand{\Id}{\mathrm{I}}
\newcommand{\Idmat}{\mathbb{I}}
\newcommand{\restr}{\Bigr|}%restriction

% names
\newcommand{\loc}{ \mathrm{loc} }

% probability theory
\newcommand{\borel}{\mathcal{B}}
\newcommand{\law}{\mathrm{law}}
\newcommand{\Exp}[1]{\mathbb{E}\left[ #1 \right] }
\newcommand{\Var}[1]{ \mathrm{var} \left( #1 \right) }
\newcommand{\Cov}[1]{ \mathrm{Cov}\left(#1 \right) }
\newcommand{\given}{ \, \mid \, }
\newcommand{\Condexp}[1]{\mathbb{E}^{#1}}
\newcommand{\deq}{ \overset{d}{=} }
\newcommand{\iidsim}{\overset{\mathrm{iid}}{\sim}}
\newcommand{\normal}{\mathcal{N}}
\newcommand{\laplace}{\mathrm{Lap}}
\newcommand{\PP}{\mathbb{P}}
\newcommand{\QQ}{\mathbb{Q}}
\newcommand{\Prob}{\mathcal{P}}
\newcommand{\Meas}{\mathcal{M}}

%make multiline boxed environment
\usepackage{environ}
\usepackage[ntheorem]{empheq}

\NewEnviron{mybox}[1]{%
	\begin{empheq}[box=\fbox]{#1}
	\BODY
	\end{empheq}
}

%norms and inner products
\newcommand{\prox}[2]{\text{prox}_{#1}^{#2}}


%
% --------------- BEGIN DOCUMENT
%


\begin{document}

\maketitle

\section*{Notation}

\begin{itemize}
\item $\indicator{C}$ denotes the indicator function,
\begin{align*}
\indicator{C}(x) := \begin{cases}
1, & \text{if } x \in C, \\
0, & \text{otherwise}.
\end{cases}
\end{align*}
\item $\chi_{C}$ denotes the characteristic function,
\begin{align*}
\chi_C(x) := \begin{cases}
0, & \text{if } x \in C, \\
\infty, & \text{otherwise}.
\end{cases}
\end{align*}
\end{itemize}


\section{Introduction}

We consider linearly constrained multivariate normal distributions on $\R^d$. Such distributions have the general form

\begin{align*}
p(\bm x) & \propto \exp \left( - \frac{1}{2} (\bm x - \bm m)^\top \bm \Sigma^{-1} (\bm x - \bm m) \right) \indicator{\mathcal C}(\bm x), \\
\mathcal C &= \Set{\bm x \in \R^d}{\bm A \bm x = \bm b, \bm C \bm x \geq \bm d, \bm l \leq \bm x \leq \bm u}.
\end{align*}

Sampling from this distribution is important in a range of applications, in particular in Bayesian inverse problems with constraints. In such applications, the dimension $d$ will often be very large, and the covariance matrix $\bm \Sigma$ will sometimes be ill-conditioned. In that case, sampling from $p$ using non-specialized methods does not work.

One method that was proposed for sampling from ill-conditioned log-concave distributions such as these is the so-called \textbf{proximal Markov chain Monte Carlo method} (proximal MCMC) \cite{Per16, DurMouPer18}.


\section{Description of proximal MCMC}

\subsection{Basic terminology and definitions}

In the following, let $p: \R^d \to [0, \infty)$ be a probability density function of the form
\begin{align*}
p(\bm x) = \frac{1}{Z} e^{-f(\bm x)},
\end{align*}
where $Z > 0$ is a constant and $f: \R^d \to (-\infty, \infty]$ is a convex function.

Given $\lambda > 0$, we define the \emph{proximal operator} $\prox{f}{\lambda}: \R^d \to \R$ associated to $f$ and $\lambda$ as
\begin{align*}
\prox{f}{\lambda}(\bm x) = \argmin_{\bm z} \lbrace f(z) + \frac{1}{2 \lambda} \norm{z - x}_2^2 \rbrace.
\end{align*}
Furthermore, we define the \emph{Moreau-Yosida envelope} $f_\lambda$ associated to $f$ and $\lambda$ as
\begin{align*}
f_\lambda(\bm x) = \inf_{\bm z}  \lbrace f(\bm z) + \frac{1}{2 \lambda} \norm{z - x}_2^2 \rbrace.
\end{align*}

\subsection{High-level description}

\subsubsection{Px-MALA}

The proximal Metropolis-adjusted Langevin algorithm (Px-MALA) is a special case of the general Metropolis-Hastings scheme. This means it creates a Markov chain $(\bm x_1, \ldots, \bm x_N)$ by repeating the following two steps for $n=0,\ldots, N$ (the first sample $x_0$ is given):
\begin{enumerate}
\item Given the current iterate $\bm x_n \in \R^d$, create a proposal $\bm y_{n+1} \sim q(\cdot | \bm x_n)$ from the proposal kernel $q(\cdot | \cdot)$.
\item Compute the Hastings-ratio
\begin{align*}
r_{n+1} = \min \left(1, \frac{p(\bm y_n) q(\bm x_n| \bm y_n)}{p(\bm x_n) q(\bm y_n| \bm x_n)} \right).
\end{align*}
Then, with probabilty $r$, set $\bm x_{n+1} = \bm y_n$. Otherwise, set $\bm x_{n+1} = \bm x_n$.
\end{enumerate}

Proximal MCMC uses the following proposal kernel:
\begin{align*}
q(\bm y | \bm x) =  \normal\left( \prox{f}{\delta/2}(\bm x), \delta \Idmat_d \right).
\end{align*}
That is, given $\bm x \in \R^d$, we can generate a new proposal $\bm y$ by sampling $\bm z \sim \normal(0, \delta \Idmat)$ and then setting
\begin{align*}
\bm y = \prox{f}{\delta / 2}(\bm x) + \bm z.
\end{align*}

\subsubsection{MYULA}

Alternatively, let
\begin{align*}
f(\bm x) = h(\bm x) + g(\bm x),
\end{align*}
where we assume that $h$ is twice continuously differentiable. In this case, we define the so-called Moreau-Yosida Unadjusted Langevin Algorithm (MYULA) by the iteration
\begin{align*}
& z_{n+1} \sim \normal(0, 2 \gamma_{n+1} \Idmat), \\
& x_{n+1} = x_n - \gamma_{n+1} \left(\nabla h(x_n) + \lambda^{-1}(x_n - \prox{g}{\lambda}(x_n) \right) + z_{n+1}, \\
\end{align*}
An important caveat for the samples obtained from MYULA is that they represent a smoothed distribution
\begin{align*}
p_\lambda(\bm x) = \frac{1}{Z} e^{-h(\bm x) + g_\lambda(\bm x)}.
\end{align*}
Hence, when estimating a quantity of interest
\begin{align*}
\bar \phi = \int \phi(\bm x) p(\bm x) \d x,
\end{align*}
instead of estimating it by an equally weighted sum over the samples,
\begin{align*}
\bar \phi^\text{bad} = \frac{1}{N} \sum_{n=1}^N \phi(x_n),
\end{align*}
one should use importance sampling,
\begin{align*}
\bar \phi^\text{good} = \sum_{n=1}^N w_n \phi(x_k),
\end{align*}
where
\begin{align*}
& w_n := \frac{\gamma_n e^{\bar g_\lambda(\bm x_n)}}{\sum_{m=1}^N \gamma_m e^{\bar g_\lambda(\bm x)}}, \\
& \bar g_\lambda(\bm x) = g_\lambda(\bm x) - g(\bm x).
\end{align*}

\subsection{Specialization to constrained Gaussians}

\subsubsection{Px-MALA}

In our particular case, we have
\begin{align*}
f(\bm x) = \frac{1}{2}(\bm x - \bm m)^\top \bm \Sigma^{-1} (\bm x - \bm m) + \chi_{\mathcal C}(\bm x).
\end{align*}
In order to evaluate the proximal operator $\prox{f}{\lambda}$, one has to solve a constrained least-squares problem
\begin{align*}
\prox{f}{\lambda}(\bm x) = \argmin_{\bm z} \Set{\norm{\bm \Sigma^{-1/2}(\bm z - \bm m)}_2^2 + \frac{1}{\lambda} \norm{\bm z - \bm x}_2^2}{\bm A \bm z = \bm b, \bm C \bm z \geq \bm d, \bm l \leq \bm z \leq \bm u}.
\end{align*}
This problem can be solved fast using any method for quadratic optimization.

\subsubsection{MYULA}

For MYULA, we use the following splitting:
\begin{align*}
& f(\bm x) = h(\bm x) + g(\bm x), \\
& h(\bm x) = \frac{1}{2} (\bm x - \bm m)^\top \bm \Sigma^{-1} (\bm x - \bm m), \\
& g(\bm x) = \chi_{\mathcal C}(\bm x), \\
& \mathcal C = \Set{\bm x \in \R^d}{\bm A \bm x = \bm b, \bm C \bm x \geq \bm d, \bm l \leq \bm x \leq \bm u}.
\end{align*}
Performing a MYULA-step thus requires evaluation of
\begin{align*}
\nabla h(\bm x) = \Sigma^{-1}(\bm x - \bm m)
\end{align*}
and
\begin{align*}
\prox{g}{\lambda}(\bm x) = \argmin_{\bm z} \Set{\norm{\bm z - \bm x}_2^2}{\bm A \bm x = \bm b, \bm C \bm x \geq \bm d, \bm l \leq \bm x \leq \bm u}.
\end{align*}
(since minimizing $\frac{1}{\lambda} \norm{\bm z - \bm x}$ is equivalent to minimizing $\norm{\bm z - \bm x}$). The latter can again be implemented using any general-purpose quadratic optimization solver.

\subsection{Choosing the hyper-parameters}

Finally, let us provide some guidelines on how to choose the hyperparameters for Px-MALA and MYULA:

\subsubsection*{Px-MALA}

The stepsize parameter $\delta$ should usually be small. A common rule-of-thumb is that one should choose $\delta$ such that the acceptance frequency is around $0.5$. Our implementation provides an option to automatically adjust $\delta$ during burnin.

\subsubsection*{MYULA}

The sequence of stepsizes $(\gamma_n)_{n=1}^\infty$ should be nonincreasing and satisfy $\gamma_n < \frac{2}{\mathrm{Lip}(h) + 1/\lambda}$ for all $n \in \N$, where $\mathrm{Lip}(h)$ is the Lipschitz constant of $\nabla h$. If $\mathrm{Lip}(h)$ is not known, $\gamma$ has to be tuned by hand. Since in any case $\mathrm{Lip}(h) \geq 0$, we now that $\gamma$ and $\lambda$ in any case have to satisfy the relation $\gamma \leq 2 \lambda$. Hence, in absence of further information, $\gamma = \lambda$ can be used as a default choice.

\subsection{Pseudocode}

\subsubsection*{Px-MALA}

Note that the proposal kernel for Px-MALA satisfies
\begin{align}
\log q(\bm y | \bm x) = - \frac{1}{2\delta}\norm{\bm y - \prox{f}{\delta/2}(\bm x)}_2^2. \label{eq:pxmala_log_proposal}
\end{align}
Note that for the evaluation of the Hastings ratio, it is easier to work with log-probabilities. That is, instead of computing
\begin{align*}
r = \min \left( 1, \frac{p(\bm y) q(\bm x | \bm y)}{p(\bm x) q(\bm y | \bm x)} \right),
\end{align*}
it is better to compute (using \eqref{eq:pxmala_log_proposal})
\begin{align*}
s = h(\bm x) - h(\bm y) + \frac{1}{2 \delta}\norm{\bm x - \prox{f}{\delta/2}(\bm y)}_2^2 - \frac{1}{2 \delta}\norm{\bm y - \prox{f}{\delta/2}(\bm x)}_2^2
\end{align*}
and then to set $r = \min(1, e^s)$.

\begin{algorithm}[H]
\caption{\texttt{Px-MALA}}
Given a feasible point $\bm x_0 \in \R^d$, $\delta > 0$, and sample size $N \in \N$;
\begin{algorithmic}[2]\label{alg:adaptive_eki}
\For{$n=0, 1,2,\ldots, N$}
\State $\bm x \gets \bm x_n$;
\State Solve
\begin{align*}
\min_{\bm \xi} & \quad \lbrace \norm{\bm \Sigma^{-1/2}(\bm \xi - \bm m)}_2^2 + \frac{2}{\delta} \norm{\bm \xi - \bm x}_2^2 \rbrace \\
\text{s. t.} & \quad \bm A \bm \xi = \bm b, \quad \bm C \bm \xi \geq \bm d, \quad \bm l \leq \bm \xi \leq \bm u,
\end{align*}
and store the minimizer as $\bm \xi$;
\State Sample $\bm z \sim \normal(0, \delta \Idmat_d)$;
\State Set $\bm y \gets \bm \xi + \bm z$;
\State Solve
\begin{align*}
\min_{\bm \zeta} & \quad \lbrace \norm{\bm \Sigma^{-1/2}(\bm \zeta - \bm m)}_2^2 + \frac{2}{\delta} \norm{\bm \zeta - \bm y_{n+1}}_2^2 \rbrace \\
\text{s. t.} & \quad \bm A \bm \zeta = \bm b, \quad \bm C \bm \zeta \geq \bm d, \quad \bm l \leq \bm \zeta \leq \bm u,
\end{align*}
and store the minimizer as $\bm \zeta$;
\State $h \gets \frac{1}{2} \norm{\bm \Sigma^{-1/2}(\bm x - \bm m)}_2^2$;
\State $\tilde h = \frac{1}{2} \norm{\bm \Sigma^{-1/2}(\bm y - \bm m)}_2^2$;
\State $q \gets \frac{1}{2 \delta}\norm{\bm x - \bm \xi}_2^2$;
\State $\tilde q \gets \frac{1}{2 \delta}\norm{\bm y - \bm \zeta}_2^2$;
\State $s \gets h - \tilde h + q - \tilde q$;
\State $r \gets \min(1, e^s)$;
\State Sample $\eta \sim \mathrm{U}([0,1])$;
\If{$r \geq \eta$}
\State $\bm x_{n+1} \gets \bm y$;
\Else
\State $\bm x_{n+1} \gets \bm x$;
\EndIf
\EndFor
\State return $\bm x_1, \ldots, \bm x_N$;
\end{algorithmic}
\end{algorithm}


\subsubsection*{MYULA}




\section{Indicators of convergence}

Assessing the quality of an MCMC run is notoriously difficult. We present a couple of numerical quantities that allow the user to judge whether the obtained samples represent the probability distribution of interest.

\subsection{$\hat R$}

\subsection{Effective sample size}






\addcontentsline{toc}{section}{Bibliography}

\printbibliography

\end{document}
